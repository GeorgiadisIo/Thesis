\chapter*{\cseextabstract}
\addstarredchapter{\cseextabstract} % minitoc
\makecseextabstract

Το φαινόμενο των ψευδών ειδήσεων στις διαδικτυακές πλατφόρμες κοινωνικής δικτύωσης έχει δημιουργήσει διεπιστημονικές κατευθύνσεις έρευνας που προσπαθούν να επιτύχουν αυτοματοποιημένους μηχανισμούς για τον έγκαιρο εντοπισμό και περιορισμό των ψευδών ειδήσεων όπως και την αποτροπη εκτεταμένων επιπτώσεών τους στην κοινή γνώμη. Ενώ ένα μεγάλο μέρος της προηγούμενης έρευνας επικεντρώθηκε στον εντοπισμό ψευδών ειδήσεων με βάση το περιεχόμενό τους, το οποιο εκμεταλευετε χαρακτηριστικα οπως το ύφος γραφής της ιστορίας, τη στάση των εμπλεκόμενων αντιδράσεων σε αυτήν, τη γλωσσική ανάλυση. Επισης, υπαρχουν προσσεγγισεις οι οποιες εξεταζουν το σχετικό πλαίσιο οπως η εκμετάλλευση της εμπλοκής των χρηστών και της φήμης τους εντός της πλατφόρμας κοινωνικής δικτύωσης. Πολλες απο τις παραπανω τεχνικες βασίζονται κυρίως σε τεχνικές με δυνατότητα τεχνητής νοημοσύνης οι οποιες παρουσιαζουν καποια μειονεκτηματα οσον αφορα την λυση που παρεχουν στο ζητημα τον ψευδων ειδησεων. Μια αλλη ενδιαφερουσα προσεγγιση για την παροχή στρατηγικών προληπτικής παρέμβασης, ειναι τεχνικες οι οποίες βασίζονται κυρίως στην ανάλυση των χωροχρονικών χαρακτηριστικών της εξελισσόμενης ιστορίας εντός της υποκείμενης υποδομής δικτύου διάδοσης. Οι περισσότερες από αυτές τις εργασίες επικεντρώνονται κυρίως στην ανάλυση των χρονοσειρών των αντιδράσεων στις ιστορίες. Ορισμένες πρόσφατες εργασίες επικεντρώνονται στα δομικά χαρακτηριστικά του δικτύου διάδοσης. Για παράδειγμα, έχει παρατηρηθεί πειραματικά ότι μια τυπική ψευδή είδηση εξελίσσεται ταχύτερα, βαθύτερα και μακρύτερα από μια τυπική αληθινή ιστορία, εντός της πλατφόρμας του κοινωνικού δικτύου.


Στην παρούσα διατριβή συνεχίζουμε τη γραμμή της έρευνας που επικεντρώνεται στα δομικά χαρακτηριστικά του υποκείμενου δικτύου διάδοσης, το οποίο προκύπτει από τις υποθέσεις εργασίας που εισάγουμε για την συμπεριφορά των χρηστών. Η πρώτη μας διαφοροποίηση από τη βιβλιογραφία είναι ότι υιοθετούμε ένα πιθανοθεωριτικό μοντέλο για τη δημιουργία ιστοριών, στο οποίο κάθε ιστορία δημιουργείται είτε από έναν ειδικό και γίνεται αντιληπτή ως \emph{αληθής ιστορία}, είτε από κάποιον προπαγανδιστή και στη συνέχεια γίνεται αντιληπτή ως \emph{ψευδή ιστορία}. Οι ειδικοί έχουν μεγάλη πιθανότητα να δώσουν τη σωστή απάντηση στο ερώτημα που θέτει η ιστορία, επειδή για παράδειγμα βασίζονται σε συγκεκριμένα επιχειρήματα και επιστημονικά στοιχεία. Οι προπαγανδιστές, από την άλλη πλευρά, απλώς προσπαθούν να προωθήσουν μια συγκεκριμένη στάση, υπέρ ή κατά ενός πραγματικού γεγονότος με την ιστορία, ανεξαρτήτως της πραγματικής αλήθειας. Θα πρέπει να σημειωθεί ότι τόσο ένας εμπειρογνώμονας όσο και ένας προπαγανδιστής μπορεί να δώσουν είτε μια σωστή είτε μια λανθασμένη απάντηση, αλλά ο εμπειρογνώμονας είναι πολύ πιθανό να είναι σωστός.


Για το προαναφερθέν πιθανοθεωριτικό μοντέλο υπάρχει ήδη μια ανάλυση για μια πολύ απλοποιημένη περίπτωση στην οποία το υποκείμενο δίκτυο διάδοσης είναι ένα απλό κατευθυνόμενο μονοπάτι. Στην παρούσα διατριβή παρέχουμε μια παρόμοια ανάλυση για την περίπτωση στην οποία το υποκείμενο δίκτυο διάδοσης είναι ένα ριζωμένο κατευθυνόμενο δέντρο. Η μετάβαση από την απλούστερη περίπτωση στην περίπτωση των δενδρικών δικτύων, αποτελεί πρόκληση καθώς δεν ισχύει πλέον η διαδοχική φύση των αντιδράσεων των χρηστών σε μια αναδυόμενη ιστορία και οι άμεσες συνέπειες των δικών τους ενεργειών σε ολόκληρη την ιστορία. Ένα σημαντικό χαρακτηριστικό του μοντέλου, το οποίο θεωρούμε ότι προσεγγίζει το γενικότερο πρόβλημα καλύτερα από άλλες εργασίες, είναι το γεγονός ότι λαμβάνετε υπ' όψιν κάποιου είδους οικονομικών παραγόντων κατά την εξέλιξη του φαινόμενου.


Αρχικά παρέχουμε μια προσεκτική ανάλυση της συμπεριφοράς των χρηστών κατά την εξέλιξη της ιστορίας, υποθέτοντας ότι συμπεριφέρονται ορθολογικά, δηλαδή επιδιώκουν την μεγιστοποίηση της αναμενόμενης ωφέλειας με βάση τις δικές τους προηγούμενες και μεταγενέστερες πεποιθήσεις για την τιμή της βασικής αλήθειας και για τον τύπο της ιστορίας (αληθής ή ψευδής). Με βάση τις υποθέσεις εργασίας που εισάγουμε για τους χρήστες, παραθέτουμε κάποιες παρατηρήσεις για την εξέλιξη της διαδικασίας μετάδοσης μιας είδησης σε ένα τέτοιο δίκτυο Στη συνέχεια, προχωρούμε με τη συμμετοχή και της πλατφόρμας, ως ανεξάρτητου παρατηρητή ολόκληρου του δικτύου διάδοσης. Στόχος μας είναι να καθορίσουμε έναν αποτελεσματικό μηχανισμό για την πλατφόρμα, ώστε να αποφασίζει σε πραγματικό χρόνο αν και πότε ακριβώς θα παρέμβει στην εξέλιξη μιας αναδυόμενης ιστορίας, παρατηρώντας μόνο στο υποκείμενο δέντρο διάδοσης. Τέλος, παραθέτουμε φραγματικές τιμές για την αναμενόμενη ωφέλεια της πλατφόρμας για την προσέγγιση του εντοπισμού του βέλτιστου χρόνου παρέμβασης.



\noindent 

\bigskip

