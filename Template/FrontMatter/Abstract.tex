\chapter*{\abstractname}
\addstarredchapter{\abstractname} % minitoc
\makecseabstract

The proliferation of fake news in online social media platforms has opened up novel, multidisciplinary directions of research trying to achieve automated mechanisms for the timely identification and containment of fake news, and mitigation of its widespread impact on public opinion. While much of the earlier research was focused on identification of fake news based on its contents (e.g., writing style of the story, stance of involved reactions to it, linguistic analysis, etc.), or on the related context (e.g., exploitation of users’ engagement and their reputation within the social media platform, etc.), which are mostly based on AI-enabled techniques, there has been a rising interest in the provision of proactive intervention strategies which are mostly based on the analysis of the spatio-temporal characteristics of the evolving story within the underlying propagation network infrastructure. Most of these works mainly focus on the analysis of the time-series of the reactions to the stories. Some recent works focus on the structural characteristics of the propagation network. For example, it has been experimentally observed that a typical fake-news story evolves faster, deeper and farther than a typical true-story, within the social network platform.


In this thesis we continue the line of research focusing on the structural characteristics of the underlying propagation network. Our first differentiation from the literature is that we adopt a probabilistic model for the creation of stories, in which each story is created either by an expert (and is perceived as a \emph{true story}, or by some propagandist (and is then perceived as a \emph{fake story}). Experts have a high probability of providing the correct answer to the question posed by the story, e.g., because they are based on concrete arguments and scientific evidence. Propagandists, on the other hand, simply try to promote a particular stance (in favor of, or against the ground-truth answer) with the story, irrespective of the ground-truth. It should be noted that both an expert and a propagandist might provide either a correct or a false answer, but the expert is highly likely to be correct.


The above mentioned probabilistic model was proposed by Papanatasiou (2019), and was then studied and analyzed for a very simplified case in which the underlying propagation network is a simple directed path. In this thesis we provide a similar analysis for the case in which the underlying propagation network is a rooted directed tree. This is a much more challenging case, since the sequential nature of the users' reactions to an emergent story (and the direct consequences of their own actions to the entire story) no longer holds.


We first provide a careful analysis of the users' behavior during the evolution of the story, assuming that they behave rationally, i.e., they are expected-utility maximizers based on their own prior and posterior beliefs for the ground-truth value and for the type (true/fake) of the story. We then proceed with the involvement also of the platform, as an independent observer of the entire propagation network. Our goal is to determine an efficient mechanisms for the platform in order to decide in real-time whether and when exactly to intervene the evolution of an emerging story, while only observing in the underlying propagation tree.  

\noindent 

\bigskip

