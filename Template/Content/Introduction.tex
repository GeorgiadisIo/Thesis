\chapter{Introduction}
\label{ch:Introduction}


\section{Motivation}
\label{sec:Motivation}
Social media has become an important part of our daily interactions due to its easy accessibility for users. According to \footnote{\url{https://datareportal.com/reports/digital-2021-april-global-statshot}} the user base of social media such as Facebook, YouTube, Twitter and Reddit is doubled since 2015. Hence, social media has become a massive hub for information sharing and many users choose to consume news from social media platform such the above mentioned. This ease of access to social media platforms accompanied with the ability to publish information in form of news article, which is given to regular users as well, creates the phenomenon of misinformation spreading. Most recent important cases of fake news, that brought spotlight to this problem, are the U.S. presidential elections of 2016 \footnote{\url{https://news.stanford.edu/2017/01/18/stanford-study-examines-fake-news-2016-presidential-election/}} and similar incidents in Germany's election of 2017\footnote{\url{https://www.theguardian.com/world/2017/jan/09/}}. 

Researches on fake news and rumor propagation attract the academic community. There is a collection of surveys that provide an overview of the problem, techniques and challenges, such as ~\cite{Zubiaga_2018,sharma2019combating,TrueFalseNewsOnline}. Approaches on fake news detection and mitigation might vary, but there are two core concepts that are common. First, there should be a way to describe and formulate human interactions and how they share information in their ecosystem, such as a social media platform. The second core concept is based on the above formulation, that describes those interactions. Based on these formulas, an approach should devise efficient methods that detect, mitigate or even prevent the spread of rumors and fake news.

There are many interesting challenges related to the topic of fake news and rumor propagation. First and foremost, the human nature that is hard to describe or formulate for a system to process it. Understanding human behavior on the topic of fake news is important in order to improve algorithms or other ways of automation in order to prevent this phenomenon. There is a plethora of researches, similar to ~\cite{FAKEnewsCovidERA,TrueFalseNewsOnline,ImpliedTEpennycock} that provide us with hints and information in order to understand human behavior on fake news detection. A second challenge is the motive behind the spread of misinformation. The reasons might be financial, political or even based on satire. This challenge is similar to that which concerns human behavior but we specify this explicitly because the development of such model rely heavily on those motives. Such an example is YouTube where users acquire advertisement revenue based on number of views. An attractive video that contains rumors, or misinformation in general, increases the income that it generates. The fact that more users have access to social media platforms creates another issue, that is the classification of rumors. The amount of posts shared within those networks is hard to monitor and classify in order to be used for training in machine learning models. Many platforms rely on fact checking from professional journalist, that specialize on this domain. 

On the topic of fake news detection, there are several techniques from different perspectives that deal with the phenomenon of fake news detection and mitigation. One of the most common models used to deal with fake news is the epidemic model. Epidemics tend to describe precisely the propagation of fake news inside social networks because of the similarities they have concerning structure of such networks as well the propagation dynamics. Some notable researches that refine and adjust the basic epidemic models presented in ~\cite{Kleinberg}, are ~\cite{SEIZ,SIHR_probGeneratingFunct,EpidemicMeanField,SimpleSIHR,VirusWithProfiles,NetworkTopologyEpidemics}. The drawback with epidemics is that they are time inefficient since they depend on observing the rates at which the population transition occurs between different states. Aside from propagation analysis, there are linguistics-based techniques that use the content of the information in order to detect fake news. Those approaches can be effective in some cases but they suffer from the fact that most of the time we do not have the exact values of ground truth in order to train those models ~\footnote{Models that do linguistic analysis are leveraging machine learning models.}. 


\section{Objectives}
\label{sec:Objectives}
Propagation analysis seems prominent approach in order to solve the problem of detecting fake news in online social media platforms. An interesting model is provided in ~\cite{papanastasiou} which is a sequential model that consists of a network of agents and a platform that monitors behaviors in that network ~\footnote{We provide every detail that we use from ~\cite{papanastasiou} but we strongly suggest the reader to study for deeper insight.}. Although social networks in real applications are by far more complex, the philosophy of this sequential model can be extended to more complex case studies. Our main objective in this thesis is to improve and adjust this model in order to work for tree propagation networks which is more representative version of a real world scenario and mitigate the problem faster than the approaches in related literature.

This modification comes with challenges concerning complexity of calculations that arise from the fact that social networks are complex structures. For this challenge, we assume in this thesis that the network we are working on is an $m$-ary tree, which is a more realistic representation of a social structure than the sequential model based on paths provide. This transition form a simple path to an $m$-ary tree comes with challenges such the formulation of propagation dynamics and the complexity of calculations from platform's side. We deal with this issue by providing the appropriate assumptions. First and foremost, we assume that platform, which can be seen as an \textit{super} agent, possess some distributional information about the other agents' prior beliefs of the story's type (true or fake) but they otherwise have access only to the events revealed to them by the structure of the propagation network. For example, neither an agent nor the platform may actually know what another agent truly believes about the evolving story. They can only observe that this agent, transmitted the story to its followers, but not the reasoning of an action (e.g., she could blindly transmit the story, or she might have conducted a private fact-check and then realized that the story is true). To simplify the analyses in our model, we also assume that this is a given Gaussian distribution. Another assumption that helps us deal with complexity, is the knowledge that each entity posses throughout the process. Those two assumptions make a natural transition from a sequential model that works for paths, to a more general model that represents tree propagation networks.

Another objective that we have in this thesis is that our model assumed to work under uncertainty. As we already mentioned in the previous paragraph, we make assumptions for the knowledge that agents and platform possesses. This is very important for two reasons. First, it makes our model more general. Reducing the amount of knowledge each entity posses makes the model more general and can work in many scenarios. The second reason is that it respects privacy of personal information and the opinions of agents fall under that category. There are many regulations such as the European Union general data protection regulation that protect personal information and many social media platforms are taking precautions in order to adjust to those regulations. By limiting the amount of knowledge to a platform, we can have such models that can be used in real life scenarios.



\section{Structure}
\label{sec:Structure}
This thesis consists of five chapters and it is structured as follows. In chapter ~\ref{ch:Preliminaries} we provide background for two basic topics, branching processes on trees and Bayesian inference, that will be mentioned and used extensively in the analysis of our news-propagation model. Chapter ~\ref{ch:Agents Propagation Model}  provides the details of the proposed news-propagation model followed by an analysis of agent's dynamics of the news sharing process with the appropriate propositions and lemmas. In chapter ~\ref{ch:Platforms Problem} we formulate the platform's dynamics and we describe our solution for the optimal inspection time. Finally, in chapter {put conclusion chapter} we have a discussion on how to generalize the model in more realistic structures followed by our concluding remarks.